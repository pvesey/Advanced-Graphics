\input{../HeadMatter.tex}
	
\part*{Assignment 3 – Domestic Property with  Revit, Enscape \& V-Ray }

\begin{tabularx}{\textwidth}{ |X|X| }
	\hline
	\textbf{Issue Date:} & 16$^{th}$ March 2021 \\
	\hline 
	\textbf{Submission Date:}  & 30$^{th}$ April 2021  \\
	\hline
\end{tabularx}


\section*{Continuous Assessment Marks}
This assignment will account for 30\% of the 100\% allocated for continuous assessment in this module

\section*{Assignment Outline}
In this assignment you will create a number of renders using Revit, Enscape and V-Ray.  You will also create a professional presentation video using Enscape and a video editor of your choice.  The building is the same as that used for Assignment 2. \\

You are to create the following:

\begin{enumerate}
	\item 1 V-Ray Render, with post production in Photoshop or similar
	\item 120 second Enscape Video
	\item 2 External Renders, one using V-Ray and one using Enscape
	\item Solar Study of the living area
	\item 3 Technical Animations
\end{enumerate}

We will be working through the production processes of this work over the course of the next few weeks.\\

\newpage

\section*{V-Ray Render}

You are required to produce 1 high quality photo-realistic render using V-Ray.  Post production should be carried out using an editor of your choice.  I recommend either Photoshop or Krita.

You should also make full use of Depth of Field, Dodge and Burn, and other tools available to enhance your image.  Adjust image attributes such as Brightness, Saturation, and Color Temperature in order to produce high quality images.

\subsection*{File format and specifications}

\begin{tabularx}{\textwidth}{ |X|X|X| }
	\hline
	\textbf{File Type:} & file extension & additional information\\
	\hline 
	Image Files  & .png & 1920 x 1080 px \\
	\hline
\end{tabularx}





\section*{120 Second Enscape Video}

You are required to produce a 120 second video animation to present the building.  You animation should comprise of several video clips edited as necessary to present a professional video presentation.  Your video should include opening credits, captions and closing credits, as well as an audio track.  Ensure that your video transitions match the beats/transition points of your audio track

Make sure that you plan out your animation. A common approach is to sequence your clips in order to best convey your message.  An example is:

\begin{itemize}
	\item Fade from Black
	\item Opening Credits
	\item Transition
	\item Establishment (wide angle)
	\item Subject Shot 
	\item Detail Shot
	\item ...(repeat)
	\item Closing 
	\item Fade to Black
	\item Credits
\end{itemize}




\subsection*{File format and specifications}

\begin{tabularx}{\textwidth}{ |X|X|X| }
	\hline
	\textbf{File Type:} & file extension & additional information\\
	\hline 
	Video File  & .mp4 & 1920 x 1080 px \\
	\hline
\end{tabularx}



\section*{External Renders}

You are required to produce 2 external renders of the building.  These renders should include a suitable spherical image to act as a backdrop to the building.  Where possible you should use the spherical image for Image Based lighting.  Pay particular attention to reflections on the building glass to ensure that the building appears as part of the landscape.\\

One image is to be created with Enscape, and the other image is to be created with V-Ray.  Both images will have the same final resolution of 1920 x 1080px

\subsection*{File format and specifications}

\begin{tabularx}{\textwidth}{ |X|X|X| }
	\hline
	\textbf{File Type:} & file extension & additional information\\
	\hline 
	Enscape Render  & .png or .jpg & 1920 x 1080px \\
	V-Ray Render & .png or .jpg & 1920 x 1080px \\
	\hline
\end{tabularx}



\section*{Solar Study of Living Space}

Use Revit to create a solar study of the living space of the building. Set the date to the summer solstice and frame rate to 1 fps, with frames at 15 minute intervals.  Ensure you select a good camera angle that will capture the lighting and shadow effects of the study area.  Render the animation using 'Shaded with Edges'.  Basic post production is required in a video editor to add opening and closing credits.\\

\subsection*{File format and specifications}

\begin{tabularx}{\textwidth}{ |X|X|X| }
	\hline
	\textbf{File Type:} & file extension & additional information\\
	\hline 
	Revit Solar Study  & .avi or .mp4 & 1920 x 1080px \\
	\hline
\end{tabularx}


\newpage
\section*{Technical Animations}

You are required to produce 3 'technical' animations.  These will involve creating several video clips using the same camera path.  The clips will then be edited in post production using a video editor of your choice in order to achieve the transition effects as below:

\begin{itemize}
	\item Day to Night (Enscape)
	\item White to Color (Enscape)
	\item Wireframe to Color (Revit)
\end{itemize}

Your videos should also include opening and closing credits

\subsection*{File format and specifications}

\begin{tabularx}{\textwidth}{ |X|X|X|X| }
	\hline
	\textbf{File Type:} & file extension & duration & additional information\\
	\hline 
	Video Files  & .mp4 & 7-10 seconds & 1920 x 1080 px  \\
	\hline
\end{tabularx}







\newpage

\section*{Production Method}

You should make full use of the tools available to you for the production of these images.  These include the suite of tools available within Revit and V-Ray for materials, lighting, HDR, Image channels, and post production within Photoshop.\\

As part of this assignment you are required to submit a two page report on the production methodology used for your images.  This report should include details of the functionality used. It should also detail the usage of Photoshop for compositing and any other modifications made. \\




\section*{Submission}
Upon completion, create a single zip file of your render output, completed videos and methodology report.  The files should be placed in a logical folder structure reflective of the file content and the production method.  You should also have one folder titled 'FinalRenders' where duplicates of your final images are placed.  Upload this single zip file to MS Teams on or before the submission deadline.

\subsection*{Marking Scheme}

\begin{table}[h!]
	\begin{center}
		\begin{tabular}{p{5cm}  p{5cm} }
			\toprule
			\textbf{Element} & \textbf{Proportion} \\ 
			\cmidrule(r){1-1}\cmidrule(lr){2-2}
			V-Ray Render & 20\%\\
			120 Second Video & 30\%\\
			External Renders & 15\%\\
			Solar Study & 10\%\\
			Technical Animation & 15\%\\
			Production Report (.pdf) & 10\%
			\\ \bottomrule
		\end{tabular}
		\label{tbl:markSchemeAsmt3}
	\end{center}
\end{table}




\end{document}