\input{HeadMatter.tex}
	
\begin{flushleft}
\Large\textbf{Assignment 3 - Masterpiece (30\%)}\\
\end{flushleft}

The concept of a masterpiece dates back hundreds of years.  A masterpiece is an object that demonstrates the skill and proficiency of its creator.  In this assignment you are going to demonstrate your mastery of Autodesk 3D Studio Max, and the V-Ray rendering engine by re-creating a single rendered image of a scene of your own creation.  Your creation is to be similar to those shown in Figures \ref{fig:1} to \ref{fig:6}.\\

The key deliverables of this project are therefore:

\begin{enumerate}
	\item Single 3D Studio Max, V-Ray Render (png or jpg of approx 4k resolution)
	\item 3DS Project folder with all files and assets included
	\item A 4 page Report on the creation of the image and the work-flow used.
\end{enumerate}

This project will take a considerable amount of time to complete.  Final renders may take over an hour to produce.  Success in this project will require an immediate start and at least 4 hours per week in addition to allocated class time.  \\

\textbf{Suggested Production Workflow}\\
Creation of images of the quality shown below requires planning and discipline in both time management and asset management.  I suggest you adopt a structured approach similar to that shown below:

\begin{itemize}
	\item Each component (asset) to be a separate 3DS file, XRefed into the 'Master' Scene.
	\item Create each asset fully, including all geometry and materials and run test renders to ensure each asset performs as expected
	\item Create an empty building model with a default material and light it  appropriately.  It is easier to light the scene when there are less objects within it.  
	\item Add the materials to the building scene.  Ensure all materials and lighting behave within expected parameters
	\item Populate the master scene with the building model and other assets and render.
\end{itemize}

\newpage
\textbf{Submission}\\

You are required to submit all assets and files used to create your scene.  This includes all materials, models, images, and other items generated.  Create a single zip file of your project folder and upload to Moodle on or before the submission date.\\


\input{FileSetup.tex}

\input{LateSubmission.tex}
\\

\textbf{Factors that will be considered when marking}\\
\begin{itemize}
	\item Realism of images
	\item Lighting
	\item Materials
	\item Composition
	\item Render errors
	\item Polygon count
\end{itemize}



\begin{figure}
	\centering
		\includegraphics[width=10cm]{img/blackKitchen.jpg}
	\caption{Black Kitchen by Supardiyono Ono}
	\label{fig:1}
\end{figure}



\begin{figure}
	\centering
		\includegraphics[width=8cm]{img/AH.jpg}
	\caption{Adpropeixe House by Bionic Digital}
	\label{fig:2}
\end{figure}


\begin{figure}
	\centering
		\includegraphics[width=10cm]{img/bath1.jpg}
	\caption{Stylish Bath by Aakash Prabu}
	\label{fig:3}
\end{figure}


\begin{figure}
	\centering
		\includegraphics[width=8cm]{img/8_large.jpg}
	\caption{Living and Dining Rendering by Gaurav 3d}
	\label{fig:4}
\end{figure}



\begin{figure}
	\centering
		\includegraphics[width=10cm]{img/bedroom.jpg}
	\caption{Interior Design by Chetan Chouhan}
	\label{fig:5}
\end{figure}



\begin{figure}
	\centering
		\includegraphics[width=10cm]{img/bath2.jpg}
	\caption{Bathroom Design by Uros Kovacevic}
	\label{fig:6}
\end{figure}






\end{document}