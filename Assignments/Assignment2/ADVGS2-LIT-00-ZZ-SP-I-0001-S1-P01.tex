\input{../HeadMatter.tex}
	
\part*{Assignment 2 – Domestic Property with  Revit, Enscape \& V-Ray }

\begin{tabularx}{\textwidth}{ |X|X| }
	\hline
	\textbf{Issue Date:} & 11$^{th}$ January 2021 \\
	\hline 
	\textbf{Submission Date:}  & 20$^{th}$ February 2021  \\
	\hline
\end{tabularx}


\section*{Continuous Assessment Marks}
This assignment will account for 40\% of the 100\% allocated for continuous assessment in this module

\section*{Assignment Outline}
In this assignment you will create a design, design presentation assets and a presentation board for a domestic building provided to you in the form of a Revit file. \\

You are to create the following:

\begin{enumerate}
	\item 7 Enscape static renders
	\item 3 V-Ray renders
	\item 120-180 second Enscape video
	\item A1 Presentation board of your design
	\item Documentation of your render production process.
\end{enumerate}

We will be working through the production processes of this work over the course of the next few weeks.\\

\newpage

\section*{Enscape Renders}

You are required to produce 7 renders using the Enscape visualisation engine.  Your 7 renders should include at least one of each of the following:
\begin{itemize}
	\item Polystyrol
	\item Light View
\end{itemize}

You should also make full use of Depth of Field and Focal Points.  You should also adjust image attributes such as Brightness, Saturation, and Color Temperature in order to produce high quality images.

\subsection*{File format and specifications}

\begin{tabularx}{\textwidth}{ |X|X|X| }
	\hline
	\textbf{File Type:} & file extension & additional information\\
	\hline 
	Image Files (x5)  & .png & 1920 x 1080 px \\
	Polystyrol Image & .png & 1920 x 1080 px \\
	Light View Image  & .png & 1920 x 1080 px \\
	\hline
\end{tabularx}





\section*{V-Ray Renders}

You are required to produce 3 renders, including one HDR, using the V-Ray for Revit.  As with the Enscape renders, you should also make full use of Depth of Field and other image attributes and techniques.  You should also make full use of Material Overrides available in the V-Ray Appearance Manager.  In order to create your HDR, you should create several renders at various exposures, and composite them using Adobe Photoshop.

\subsection*{File format and specifications}

\begin{tabularx}{\textwidth}{ |X|X|X| }
	\hline
	\textbf{File Type:} & file extension & additional information\\
	\hline 
	Image Files (x2)  & .jpg & 1920 x 1080 px \\
	HDR Image & .jpg & 1920 x 1080 px \\
	\hline
\end{tabularx}



\section*{A1 Presentation Board}

You are required to produce a presentation board using Adobe Photoshop.  Your board should contain at least 3 of the images that you have produced for the previous sections.  Use separate layers for each of your images, and export the final image as a pdf or jpg file.  Do not flatten the layers prior to submission.

\subsection*{File format and specifications}

\begin{tabularx}{\textwidth}{ |X|X|X| }
	\hline
	\textbf{File Type:} & file extension & additional information\\
	\hline 
	A1 Presentation Sheet  & .pdf or .jpg & 594 x 841mm\\
	Photoshop File & .psd & \\
	\hline
\end{tabularx}



\section*{Enscape Video Animation}

The runtime for your video is to be between 120 and 180 seconds.  Your video should comprise of several clips edited using Kinemaster or a similar tool.  In addition to intro outro/credit clips, you are also required to include an audio track on your video. The audio track can be added within Enscape or your video editor. \\

Make sure that you plan out your animation. A common approach is to sequence your clips in order to best convey your message.  An example is:

\begin{itemize}
	\item Fade from Black
	\item Opening Credits
	\item Transition
	\item Establishment (wide angle)
	\item Subject Shot 
	\item Detail Shot
	\item ...(repeat)
	\item Closing 
	\item Fade to Black
	\item Credits
\end{itemize}


\subsection*{File format and specifications}

\begin{tabularx}{\textwidth}{ |X|X|X| }
	\hline
	\textbf{File Type:} & file extension & additional information\\
	\hline 
	Video File  & .mp4 & 120 to 180 seconds \\
	- & - & 1920 x 1080 px \\
	Audio File & .mp3 & \\
	Intro Clip & - & 5 seconds\\
	Outro/Credits Clip & - & 20 seconds\\
	\hline
\end{tabularx}









\section*{Production Method}

You should make full use of the tools available to you for the production of these images.  These include the suite of tools available within Revit and V-Ray for materials, lighting, HDR, Image channels, and post production within Photoshop.\\

As part of this assignment you are required to submit a two page report on the production methodology used for your images.  This report should include details of the functionality used. It should also detail the usage of Photoshop for compositing and any other modifications made. \\




\section*{Submission}
Upon completion, create a single zip file of your render output, completed video, presentation board and methodology report.  The files should be placed in a logical folder structure reflective of the file content and the production method.  You should also have one folder titled 'FinalRenders' where duplicates of your final images are placed.  Upload this single zip file to MS Teams on or before the submission deadline.

\subsection*{Marking Scheme}

\begin{table}[h!]
	\begin{center}
		\begin{tabular}{p{5cm}  p{5cm} }
			\toprule
			\textbf{Element} & \textbf{Proportion} \\ 
			\cmidrule(r){1-1}\cmidrule(lr){2-2}
			Enscape Renders & 15\%\\
			V-Ray Renders & 15\%\\
			Presentation Board & 20\%\\
			Video Animation & 30\%\\
			Production Report (.pdf) & 20\%
			\\ \bottomrule
		\end{tabular}
		\label{tbl:markSchemeAsmt3}
	\end{center}
\end{table}




\end{document}