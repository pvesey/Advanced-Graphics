\input{HeadMatter.tex}
	
\begin{flushleft}
\Large\textbf{Assignment 8 - Treehouse (25\%)}\\
\end{flushleft}

This assignment is expands Project Brief 4 issued for Design Studio 3, namely the Treehouse project. \\

The key deliverables for Advanced Graphics \& Visualisation are:

\begin{enumerate}
	\item Photo-realistic Renders of key rooms and features
	\item Unity Game Engine Environment
\end{enumerate}

\textbf{Photo-realistic Renders}\\

You are required to produce 3DS Mental Ray renders of the following key areas:

\begin{enumerate}
	\item Reception Area
	\item Therapy Room for Physiotherapy,Occupational, Speech \& Language Therapy, \& other activities
	\item Family Room
	\item Circulation Areas
\end{enumerate}

You are required to produce at least one render of each of the above.  It is expected that more than one will be required in each case in order to fully showcase your design.  Please note that these renders are expected to be of portfolio standard.\\

\textbf{Unity Game Engine Environment}\\

The game engine real-time environment must be a complete model of the intended built asset.  Materials and lighting do not need to be photo-realistic, however they must be comparable with your final design.  You are advised to use the Autodesk Material Converter (AMC3200) script installed in 3DS to convert your Mental Ray materials to Standard materials prior to export to Unity.\\


\textbf{Suggested Production Workflow}\\

As with Assignment 7, the creation of images requires planning and discipline in both time management and asset management.  I suggest you adopt a structured approach similar to that shown below:

\begin{itemize}
	\item Each component (asset) to be a separate 3DS file, XRefed into the 'Master' Scene.
	\item Create each asset fully, including all geometry and materials and run test renders to ensure each asset performs as expected
	\item Create an empty building model with a default material and light appropriately.  It is easier to light the scene when there are less objects within it.  
	\item Add the materials to the building scene.  Ensure all materials and lighting behave within expected parameters
	\item Populate the master scene with the building model and other assets and render.
	\item Once the model is complete, it can be imported into Unity via an FBX file.
\end{itemize}

\textbf{Submission and Demonstration}\\

You are required to submit all assets and files used to create your scene.  This includes all materials, models, images, and other items generated.  Create a single zip file and upload to Moodle on or before the submission date.\\

You will present your render and demonstrate your game engine version to the class and other lecturing staff at a date and time as advised on Moodle.  All class members are required to attend the demonstrations.\\


\textbf{Factors that will be considered when marking}\\
\begin{itemize}
	\item Realism of images
	\item Lighting
	\item Materials
	\item Composition
	\item Render errors
	\item Polygon count
	\item Game engine 
\end{itemize}



\end{document}