\input{HeadMatter.tex}
	
%     _            _                                  _     _____ 
%    / \   ___ ___(_) __ _ _ __  _ __ ___   ___ _ __ | |_  |___ / 
%   / _ \ / __/ __| |/ _` | '_ \| '_ ` _ \ / _ \ '_ \| __|   |_ \ 
%  / ___ \\__ \__ \ | (_| | | | | | | | | |  __/ | | | |_   ___) |
% /_/   \_\___/___/_|\__, |_| |_|_| |_| |_|\___|_| |_|\__| |____/ 
%                    |___/                                        

\newpage
\setcounter{page}{1}
\begin{center}
	\begin{figure}[ht]
		\centering
		\includegraphics[width = 6cm]{img/LITlogo.jpg}
		\label{fig:logoa3}
	\end{figure}
	\Large\textbf{Practical 3 - Chair and Table}\\
	\large\textbf{BA in Interior Design \& Technology}
\end{center}
In this assignment you are going to create a simple table scene, comprising table with tablecloth, chair, and set dinner place.  You may use assets from Turbosquid and other sites.\\

The scene is to be a simple contemporary table and chair (single) overlooking or beside a window.  The external scene does not need to be modelled, but the window itself is to be a prominent feature of the final image.\\

The final image is to be of photo-realistic standard, so it will be vital to keep the scene simple, and spend considerable time on lighting and materials.\\ 

In order to complete this assignment, I suggest the following workflow.  
\begin{enumerate}
	\item Model the scene in Revit
	\item Import model to 3DS and apply materials for initial render
	\item Add other 3DS objects to the scene, and complete the modelling
	\item Create and/or modify materials within another instance of 3DS.  You should perfect your materials by testing on a small object.
	\item Apply your new materials to the final model.
	\item Test camera and lighting using a variety of test renders
	\item Create your final render with at least circa 4 bounces, final gather, GI etc.
\end{enumerate}

Your final image is to be 1920 x 1024 in dimensions.  Please note that this will take considerable time to render.

Submission is to be a zip file containing the model geometry, materials, and rendered images.\\

Upload your model to Moodle upon completion.

\end{document}