\input{HeadMatter.tex}
%    _            _                                  _     _ 
%   / \   ___ ___(_) __ _ _ __  _ __ ___   ___ _ __ | |_  / |
%  / _ \ / __/ __| |/ _` | '_ \| '_ ` _ \ / _ \ '_ \| __| | |
% / ___ \\__ \__ \ | (_| | | | | | | | | |  __/ | | | |_  | |
%/_/   \_\___/___/_|\__, |_| |_|_| |_| |_|\___|_| |_|\__| |_|
%                   |___/                                    

\newpage
\setcounter{page}{1}
\begin{center}
	\begin{figure}[ht]
		\centering
		\includegraphics[width = 6cm]{img/LITlogo.jpg}
		\label{fig:logoa1}
	\end{figure}
	\Large\textbf{Practical 1 - Revit Revision}\\
	\large\textbf{BA in Interior Design \& Technology}
\end{center}


In this practical you are required to create a simple space that we will use later in Unity 3D.  The space should consist of 4 rooms and a central corridor.  The room configuration is at your discretion.  Each room should have a small degree of furnishing and a color scheme to easily distinguish one room from another.  You will have to create \textit{inter-alia} the following model components


\begin{enumerate}
	\item Levels
	\item Floor, Walls, Ceilings and/or Roof
	\item Windows and Doors
\end{enumerate}

You do not need to model any external elements, such as toposurfaces, trees etc.  We will be adding these items later using Unity.

\end{document}





