\input{HeadMatter.tex}
	
%     _            _                                  _     _____ 
%    / \   ___ ___(_) __ _ _ __  _ __ ___   ___ _ __ | |_  |___ / 
%   / _ \ / __/ __| |/ _` | '_ \| '_ ` _ \ / _ \ '_ \| __|   |_ \ 
%  / ___ \\__ \__ \ | (_| | | | | | | | | |  __/ | | | |_   ___) |
% /_/   \_\___/___/_|\__, |_| |_|_| |_| |_|\___|_| |_|\__| |____/ 
%                    |___/                                        

\newpage
\setcounter{page}{1}
\begin{center}
	\begin{figure}[ht]
		\centering
		\includegraphics[width = 6cm]{img/LITlogo.jpg}
		\label{fig:logoa3}
	\end{figure}
	\Large\textbf{Practical 3 - Intermediate Revit to Unity}\\
	\large\textbf{BA in Interior Design \& Technology}
\end{center}

In this assignment we are going to build on what has been previously accomplished. The key objective of this exercise to create a more realistic work-flow that will allow greater flexibility during the design process.  The key deliverables are:
\begin{enumerate}
	\item Separate the building structure from the furniture and fittings.
	\item Separate animation objects from the building structure.
	\item Animate a door in Unity.
\end{enumerate}

By separating the building elements we can then make edits to specific game assets quickly, without effecting other game assets.  In the case of furniture and fittings, this allows us to explore fit-out options quickly and with minimal disruption.\\

Separating doors and windows from the building structure will allow us to animate those objects within Unity.  Doors are normally animated to the open position when the user approaches during game play.  This is achieved by creating a trigger collider in the space adjacent to the door, which is then used to 'trigger' the door animation.\\

Before uploading your game to Moodle you should demonstrate your game to your lecturer.

\end{document}